\documentclass[a4paper,11pt,onecolumn,oneside]{article}
% for url
\usepackage{hyperref}

%for draw chart
\usepackage{tikz}
\usetikzlibrary{positioning,shapes,shadows,arrows,fit}
\usetikzlibrary{backgrounds}

%for code
\usepackage{listings}

\lstloadlanguages{
         Java
          }
          %for chinese indent
\usepackage{indentfirst}
\usepackage{enumerate}
\usepackage{verbatim} 

% 设定中文环境
\usepackage{xeCJK}
\setCJKmainfont[BoldFont={SimHei}]{SimSun}
\setCJKsansfont{SimSun}
\setCJKmonofont{SimHei}

\title{Blueprint使用手记}
\author{Jet Geng}
\date{\today}
\begin{document}
\maketitle
\section{概述}
Blueprint是OSGi Service Platform Enterprise Specification 标准的一部分。很多最佳实践中也都推荐在应用程序中使用他。最近在项目中用他来发布和应用服务。感觉效果不错。所以就有了这篇使用手记!欢迎各位看官拍砖!
\section{配置环境}
我们在前期找到了两个Blueprint的实现:一个是Apache 的aries,另外一个就是Eclipse的Gemini。最后选择了Gemini。做出这样的决定出于两点考虑:
\begin{itemize}
    \item Blueprint的标准就是由Spring提出。
    \item Gemini的初始代码由Srping所捐献。
\end{itemize}
下面我就简单介绍一下环境的配置过程:
\begin{itemize}
    \item 从\url{http://eclipse.org/gemini/} 下载Gemini的合适的版本。我们采用的是1.0.0M1。并解压到路径 \emph{A}
    \item 从\url{http://static.springsource.org/downloads/nightly/milestone-download.php}下载spring-osgi-2.0.0.M1-with-dependencies。 并解压到路径\emph{B}。他里面包括了Gemini所依赖的bundle。说白了也就是SpringFramework。
    \item Eclipse中新建一个Target Platform。把上述的路径\emph{A}和\emph{B} 加入到新建的Target Platform中去。
    \item 选择新建的Target Platform为当前活动的Platform。
\end{itemize}
\section{启航}
\paragraph{个人理解Gemini就是OSGi世界中的IOC。既然是一个IOC框架,那我们就从创建一个Bean开始吧!}
\subsection{创建一个Bean}

\section{常见问题}
\subsection{打包}
\subsection{服务相关}
\section{需要提高点}

\end{document}

