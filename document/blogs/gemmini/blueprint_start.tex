\documentclass[a4paper,11pt,onecolumn,oneside]{article}
% for url
\usepackage{hyperref}

%for draw chart
\usepackage{tikz}
\usetikzlibrary{positioning,shapes,shadows,arrows,fit}
\usetikzlibrary{backgrounds}

%for code
\usepackage{listings}
\usepackage{xcolor}

\lstloadlanguages{
         Java
          }
          %for chinese indent
\usepackage{indentfirst}
\usepackage{enumerate}
\usepackage{verbatim} 

% 设定中文环境
\usepackage{xeCJK}
\setCJKmainfont[BoldFont={SimHei}]{SimSun}
\setCJKsansfont{SimSun}
\setCJKmonofont{SimHei}

\tikzstyle{abstractbox} = [draw=black, fill=white, rectangle, 
  inner sep=10pt, style=rounded corners, drop shadow={fill=black,
  opacity=1}]
\tikzstyle{abstracttitle} =[fill=white]

\newcommand{\boxabstract}[2][fill=white]{
    \begin{center}
      \begin{tikzpicture}
        \node [abstractbox, #1] (box)
        {\begin{minipage}{0.80\linewidth}
%            \setlength{\parindent}{2mm}
            \footnotesize #2
          \end{minipage}};
%        \node[abstracttitle, right=10pt] at (box.north west) {Abstract};
      \end{tikzpicture}
    \end{center}
  }

\definecolor{gray}{rgb}{0.4,0.4,0.4}
\definecolor{darkblue}{rgb}{0.0,0.0,0.6}
\definecolor{cyan}{rgb}{0.0,0.6,0.6}

\lstset{
  basicstyle=\ttfamily,
  columns=fullflexible,
  showstringspaces=false,
  commentstyle=\color{gray}\upshape
}

\lstdefinelanguage{XML}
{
  morestring=[b]",
  morestring=[s]{>}{<},
  morecomment=[s]{<?}{?>},
  stringstyle=\color{black},
  identifierstyle=\color{darkblue},
  keywordstyle=\color{cyan},
  morekeywords={xmlns,version,type}% list your attributes here
}
  

\title{Blueprint使用手记}
\author{Jet Geng}
\date{\today}
\begin{document}
\maketitle
\section{概述}
Blueprint是OSGi Service Platform Enterprise Specification 标准的一部分。很多最佳实践中也都推荐在应用程序中使用他。最近在项目中用他来发布和应用服务。感觉效果不错。所以就有了这篇使用手记!欢迎各位看官拍砖!
\section{配置环境}
我们在前期找到了两个Blueprint的实现:一个是Apache 的aries,另外一个就是Eclipse的Gemini。最后选择了Gemini。做出这样的决定出于两点考虑:
\begin{itemize}
    \item Blueprint的标准就是由Spring提出。
    \item Gemini的初始代码由Srping所捐献。
\end{itemize}
下面我就简单介绍一下环境的配置过程:
\begin{itemize}
    \item 从\url{http://eclipse.org/gemini/} 下载Gemini的合适的版本。我们采用的是1.0.0M1。并解压到路径 \emph{A}
    \item 从\url{http://static.springsource.org/downloads/nightly/milestone-download.php}下载spring-osgi-2.0.0.M1-with-dependencies。 并解压到路径\emph{B}。他里面包括了Gemini所依赖的bundle。说白了也就是SpringFramework。
    \item Eclipse中新建一个Target Platform。把上述的路径\emph{A}和\emph{B} 加入到新建的Target Platform中去。
    \item 选择新建的Target Platform为当前活动的Platform。
\end{itemize}
\section{启航}
\paragraph{个人理解Gemini就是OSGi世界中的IOC。既然是一个IOC框架,那我们就从创建一个Bean开始吧!}
\subsection{创建一个Bean}
首先我们来看一下我们要存入Container中的POJO。
\begin{lstlisting}[keywordstyle=\color{blue!70}, commentstyle=\color{red!50!green!50!blue!50} ,language=Java]
package org.gunn.gemini.demo;

import org.slf4j.Logger;
import org.slf4j.LoggerFactory;
/**
 * This pojo will create by blueprint container
 * @author Jet Geng 
 *
 */
public class POJOWillInContainer {
    private Logger logger = 
            LoggerFactory.getLogger(POJOWillInContainer.class);
    
    private String name ;
    
    private String age;

    public void setName(String name) {
        logger.info("the new name is:" + name);
        this.name = name;
    }

    public void setAge(String age) {
        logger.info("the new age value is:" + age);
        this.age = age;
    }
}
    
\end{lstlisting}
这个超级简单的一个POJO,我们如何通过Gemini来创建他呢?我们通过一个简单的配置文件。具体如下。
\begin{lstlisting}[language=Xml]
<?xml version="1.0" encoding="UTF-8"?>
<blueprint xmlns="http://www.osgi.org/xmlns/blueprint/v1.0.0">
	<bean id="myPOJO" class="org.gunn.gemini.demo.POJOWillInContainer" >
		<property name="name" value="JetGeng"/>
		<property name="age" value="32"/>
	</bean>
</blueprint>
    
\end{lstlisting}
用过spring的兄弟,第一眼就能看明白这个配置文件说的是什么。他就在
\boxabstract{
这是一个提醒
}
\section{常见问题}
\subsection{打包}
\subsection{服务相关}
\section{需要提高点}

\end{document}

